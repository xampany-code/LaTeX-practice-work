\documentclass{beamer}
\usepackage[utf8]{inputenc}
\usepackage[usenames]{color}
\usepackage{colortbl}
\usetheme{Dresden}
\usecolortheme{default}

\title{LaTeX  practice work}
\subtitle{Beamer presentation}
\author{Anna Gertsog}
\institute[HSE]{Faculty of Computer Science,\\ Data Science and Business Analytics, \\ HSE University}
\date{25 July 2022}

\logo{\includegraphics[height=1.5cm]{images/logo.png}}

\begin{document}

\frame{\titlepage}

\begin{frame}
\frametitle{Table of Contents}
\tableofcontents
\end{frame}

\section{Intro}
\begin{frame}
\frametitle{Introduction to LaTeX}
\textbf{What is LaTeX?}\\
LaTeX is the collective name for a document preparation system. It includes a set of tools that generate print-ready documents (usually in PDF format) from text files written using a special markup language. TeX itself is the low-level markup and programming language which underpins it.
\end{frame}

\begin{frame}
\frametitle{Why and what is LaTeX used for?}
Typically, LaTeX is used to write articles, documents, reports and develop presentations.\\
This markup language has a number of advantages:
\begin{itemize}
    \item Easy to type mathematical formulas, design graphs, algorithms and other formulas
    \item Style, fonts, table layout and drawings are consistent throughout the whole document
    \item It is free and easy to use
    \item Since the source document simply contains text, you can use software tools to create tables, figures, formulas, etc
\end{itemize}
\end{frame}

\section{My experience}

\begin{frame}
\frametitle{My experience of using LaTeX}

Initially, I had only heard about LaTeX from friends, who told me how useful it was for writing formulas for mathematical assignments.  However, I considered the markup language itself to be complicated, because you have to learn a number of commands to use it.\\
Eventually I realised that I wanted to learn LaTeX and I would need it in my 3rd year at \textbf{DSBA}, so I chose this course in my summer internship. To be honest, I learned it in just a couple of hours, designed my coursework report nicely, and started actively using formulas. Now I can confidently say that I  I master this markup language well and I will definitely actively use it in the future and recommend it to my friends.
\end{frame}

\section{LaTeX features}
\begin{frame}{LaTeX features}
    As already mentioned, LaTeX is as easy as possible to create formulas, which is a distinct advantage over Word, for example:\\
    \begin{equation*}
\textcolor{cyan}{\sqrt[n]{1+x+x^2+x^3+\dots+x^n}}
     \end{equation*}\\
     \begin{center}
     It's also convenient for inserting pictures:\\
       \end{center}
         \begin{center}
     \includegraphics[width=0.4\linewidth]{images/Gregory_and_watermelon.jpeg}\quad\includegraphics[width=.35\linewidth]{images/gosha.jpeg}
          \end{center}
\end{frame}
\section{Conclusion}
\begin{frame}{Conclusion}
\begin{center}
   Thank you for the course and for your attention!
    \end{center}
\end{frame}
\end{document}
